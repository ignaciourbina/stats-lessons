\documentclass{article}
\begin{document}
\begin{enumerate}
\item When sampling from a population with mean $\mu$, Lecture~10 states that the expected value of the sample mean $\bar{x}$ is
  \begin{enumerate}[label=(\Alph*)]
  \item $\mu$
  \item $\bar{x}/n$
  \item $0$
  \item $\mu/n$
  \end{enumerate}
\item According to Lecture~10, if the population is normal with known variance, which statistic standardizes $\bar{x}$?
  \begin{enumerate}[label=(\Alph*)]
  \item $t=\dfrac{\bar{x}-\mu}{s/\sqrt{n}}$
  \item $Z=\dfrac{\bar{x}-\mu}{\sigma/\sqrt{n}}$
  \item $Z=\dfrac{\bar{x}}{\sigma}$
  \item $t=\dfrac{\bar{x}}{s}$
  \end{enumerate}
\item Lecture~10 introduces the Central Limit Theorem. Which statement best reflects it?
  \begin{enumerate}[label=(\Alph*)]
  \item For large $n$, the distribution of $\bar{x}$ is approximately normal regardless of population shape.
  \item Small samples from any distribution are normal.
  \item The population mean equals the sample mean for large $n$.
  \item Variance estimates are unnecessary for large samples.
  \end{enumerate}
\item One motivation in Lecture~10 for comparing two means is
  \begin{enumerate}[label=(\Alph*)]
  \item assessing job satisfaction across industries
  \item computing a single sample proportion
  \item drawing histograms of one variable
  \item maximizing sample size
  \end{enumerate}
\item From Lecture~11, the margin of error in a large-sample confidence interval for $\mu$ is
  \begin{enumerate}[label=(\Alph*)]
  \item $z_{1-\alpha/2}\dfrac{s}{\sqrt{n}}$
  \item $\dfrac{s}{n}$
  \item $z_{1-\alpha/2}s$
  \item $\dfrac{s^2}{n}$
  \end{enumerate}
\item Using $n=100$, $\bar{x}=32.5$, and $s=7$ from Lecture~11, the standard error of $\bar{x}$ is closest to
  \begin{enumerate}[label=(\Alph*)]
  \item $0.7$
  \item $3.5$
  \item $7.0$
  \item $1.4$
  \end{enumerate}
\item With the values in Question~6, the 95\% confidence interval for the mean commute time is
  \begin{enumerate}[label=(\Alph*)]
  \item $(31.1,\,33.9)$
  \item $(30.0,\,35.0)$
  \item $(32.0,\,33.0)$
  \item $(25.0,\,40.0)$
  \end{enumerate}
\item Lecture~11 gives the large-sample confidence interval for $\mu_1-\mu_2$ as
  \begin{enumerate}[label=(\Alph*)]
  \item $(\bar{x}_1-\bar{x}_2)\pm z_{1-\alpha/2}\sqrt{\dfrac{s_1^2}{n_1}+\dfrac{s_2^2}{n_2}}$
  \item $(\bar{x}_1+\bar{x}_2)\pm z_{1-\alpha/2}(s_1+s_2)$
  \item $(\bar{x}_1-\bar{x}_2)\pm \dfrac{s_1^2+s_2^2}{n_1+n_2}$
  \item $(\bar{x}_1-\bar{x}_2)\pm t_{n_1+n_2}$
  \end{enumerate}
\item Which of the following is NOT one of the hypothesis-test steps listed in Lecture~11?
  \begin{enumerate}[label=(\Alph*)]
  \item State $H_0$ and $H_a$
  \item Choose $\alpha$
  \item Randomize the data a second time
  \item Compute a test statistic and make a decision
  \end{enumerate}
\item Lecture~12 explains that when $n<30$ and $\sigma$ is unknown, inference for a single mean uses
  \begin{enumerate}[label=(\Alph*)]
  \item the Student-$t$ distribution with $n-1$ degrees of freedom
  \item the standard normal distribution regardless of $n$
  \item bootstrapping only
  \item no distributional assumptions
  \end{enumerate}
\item For small samples, Lecture~12 gives the confidence interval formula
  \begin{enumerate}[label=(\Alph*)]
  \item $\bar{x}\pm t^*_{df}\dfrac{s}{\sqrt{n}}$
  \item $\bar{x}\pm z\dfrac{s}{n}$
  \item $\bar{x}\pm s$
  \item $\bar{x}\pm z^*_{df}\dfrac{s}{\sqrt{n}}$
  \end{enumerate}
\item In the voter-ID example of Lecture~12 with $n=10$, $\bar{d}=-1.8$ and $s_d=2.7$, the 95\% confidence interval for the mean difference is roughly
  \begin{enumerate}[label=(\Alph*)]
  \item $(-3.7,\,0.1)$
  \item $(-1.8,\,1.8)$
  \item $(0,\,3.6)$
  \item $(-5.0,\,2.0)$
  \end{enumerate}
\item When variances differ across groups, Lecture~12 recommends using
  \begin{enumerate}[label=(\Alph*)]
  \item Welch's $t$ statistic
  \item a pooled standard deviation
  \item the $z$ test
  \item the sign test only
  \end{enumerate}
\item In the civics quiz example from Lecture~12, the conclusion was that
  \begin{enumerate}[label=(\Alph*)]
  \item honors students scored significantly higher
  \item there was no difference between groups
  \item regular students scored higher
  \item results were inconclusive
  \end{enumerate}
\item Lecture~12 lists which of the following as a common trap when using two-sample $t$ methods?
  \begin{enumerate}[label=(\Alph*)]
  \item Ignoring unequal variances
  \item Setting $\alpha=0.05$
  \item Reporting the sample means
  \item Using more than two groups
  \end{enumerate}
\end{enumerate}
\end{document}
