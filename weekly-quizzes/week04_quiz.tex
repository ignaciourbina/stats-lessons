\documentclass{article}
\begin{document}
\begin{enumerate}
\item In a city poll about a new tax, $S$ out of $n$ respondents support the measure. How is the sample proportion $\hat{p}$ calculated?
  \begin{enumerate}[label=(\Alph*)]
  \item $\hat{p}=S/n$
  \item $\hat{p}=n/S$
  \item $\hat{p}=S\times n$
  \item $\hat{p}=S-n$
  \end{enumerate}
\item When repeatedly surveying voters about a candidate, what is the expected value of the sample proportion relative to the true population proportion $p$?
  \begin{enumerate}[label=(\Alph*)]
  \item $p$
  \item $\hat{p}$
  \item $np$
  \item $p/n$
  \end{enumerate}
\item In public opinion polling, which theorem allows analysts to approximate the distribution of $\hat{p}$ by a normal curve when the sample size is large?
  \begin{enumerate}[label=(\Alph*)]
  \item Central Limit Theorem
  \item Law of Large Numbers
  \item Bayes' Theorem
  \item Chebyshev's Inequality
  \end{enumerate}
\item When testing a new policy's impact, the significance level $\alpha$ represents our tolerance for which type of error?
  \begin{enumerate}[label=(\Alph*)]
  \item Type I error
  \item Type II error
  \item Sampling error
  \item Estimation error
  \end{enumerate}
\item Journalists often report a 95\% margin of error for poll results. In constructing a confidence interval, multiplying the estimated standard error by $z_{1-\alpha/2}$ yields what quantity?
  \begin{enumerate}[label=(\Alph*)]
  \item Margin of error
  \item P-value
  \item Test statistic
  \item Residual
  \end{enumerate}
\item A survey of $120$ college students finds that $66$ support a tuition freeze. Which of the following lists $\hat{p}$ and its estimated standard error $\sqrt{\hat{p}(1-\hat{p})/n}$?
  \begin{enumerate}[label=(\Alph*)]
  \item $\hat{p}=0.55$,\quad SE $\approx0.045$
  \item $\hat{p}=0.45$,\quad SE $\approx0.045$
  \item $\hat{p}=0.55$,\quad SE $\approx0.15$
  \item $\hat{p}=0.45$,\quad SE $\approx0.15$
  \end{enumerate}
\item Using the results from Question~6, where $66$ of the $120$ students favored the tuition freeze, which of the following is the $95\%$ confidence interval for $p$ computed with $\hat{p}\pm1.96\times\text{SE}$?
  \begin{enumerate}[label=(\Alph*)]
  \item $(0.46,\,0.64)$
  \item $(0.50,\,0.60)$
  \item $(0.30,\,0.80)$
  \item $(0.55,\,0.65)$
  \end{enumerate}
\item A statewide referendum poll of $400$ voters finds $220$ in favor of the measure. What is the $99\%$ confidence interval for $p$?
  \begin{enumerate}[label=(\Alph*)]
  \item $(0.49,\,0.61)$
  \item $(0.48,\,0.62)$
  \item $(0.46,\,0.64)$
  \item $(0.52,\,0.58)$
  \end{enumerate}
\item In a statewide election poll, $\hat{p}=0.52$, $p_0=0.49$, and $n=1000$. Compute the test statistic $Z$ and decide whether to reject $H_0$ at $\alpha=0.05$.
  \begin{enumerate}[label=(\Alph*)]
  \item $Z\approx1.9$, fail to reject $H_0$
  \item $Z\approx1.9$, reject $H_0$
  \item $Z\approx2.5$, reject $H_0$
  \item $Z\approx0.5$, fail to reject $H_0$
  \end{enumerate}
\item A newspaper poll reports $\hat{p}=0.53$ from a sample of $n=1000$. Test $H_0:p=0.50$ against $H_a:p>0.50$ using $Z=(\hat{p}-p_0)/\sqrt{p_0(1-p_0)/n}$. What is the one-sided $P$-value and decision at $\alpha=0.05$?
  \begin{enumerate}[label=(\Alph*)]
  \item $P\approx0.03$, reject $H_0$
  \item $P\approx0.14$, fail to reject $H_0$
  \item $P\approx0.06$, fail to reject $H_0$
  \item $P\approx0.03$, fail to reject $H_0$
  \end{enumerate}
\item Which rule of thumb from the slides ensures the sampling distribution of $\hat{p}$ is approximately normal?
  \begin{enumerate}[label=(\Alph*)]
  \item $n\hat{p}\ge10$ and $n(1-\hat{p})\ge10$
  \item $n\ge100$
  \item $p$ close to $0.5$
  \item $n\ge30$ regardless of $p$
  \end{enumerate}
\item A city survey records $245$ supporters of a bus lane out of $500$ residents. What are $\hat{p}$ and its estimated standard error?
  \begin{enumerate}[label=(\Alph*)]
  \item $\hat{p}=0.49$,\quad SE $\approx0.022$
  \item $\hat{p}=0.51$,\quad SE $\approx0.022$
  \item $\hat{p}=0.49$,\quad SE $\approx0.044$
  \item $\hat{p}=0.51$,\quad SE $\approx0.044$
  \end{enumerate}
\item Using the values from Question~12 (where $245$ of $500$ residents supported the bus lane), which option shows the $90\%$ confidence interval for $p$ computed with $\hat{p}\pm1.645\times\text{SE}$?
  \begin{enumerate}[label=(\Alph*)]
  \item $(0.45,\,0.53)$
  \item $(0.44,\,0.54)$
  \item $(0.47,\,0.51)$
  \item $(0.40,\,0.58)$
  \end{enumerate}
\item Again using the survey from Question~12 with $245$ supporters out of $500$, test $H_0:p=0.50$ against $H_a:p\ne0.50$. Which $Z$ value and decision are correct at $\alpha=0.05$?
  \begin{enumerate}[label=(\Alph*)]
  \item $Z\approx-0.45$, fail to reject $H_0$
  \item $Z\approx-0.45$, reject $H_0$
  \item $Z\approx-2.5$, reject $H_0$
  \item $Z\approx1.5$, fail to reject $H_0$
  \end{enumerate}
\item In plain language, what does a $95\%$ confidence level indicate when polls are repeated many times?
  \begin{enumerate}[label=(\Alph*)]
  \item About $95\%$ of such intervals will contain the true $p$
  \item $95\%$ of the population lies inside any one interval
  \item The chance the specific interval contains $p$ is exactly $95\%$
  \item The interval width is always $95\%$ of $p$
  \end{enumerate}
\item If a survey relies on convenience sampling instead of random sampling, which assumption of the Central Limit Theorem may be violated?
  \begin{enumerate}[label=(\Alph*)]
  \item Independence of observations
  \item Large sample size
  \item Finite variance
  \item None, the CLT always holds
  \end{enumerate}
\item Investigators want the margin of error to be no more than $0.02$ for a $95\%$ confidence level when $p$ is unknown. The slides suggest using $p=0.5$ as a conservative guess. Which sample size is closest to this recommendation?
  \begin{enumerate}[label=(\Alph*)]
  \item $n=600$
  \item $n=1200$
  \item $n=2400$
  \item $n=4800$
  \end{enumerate}
\item A poll finds a statistically significant difference in support between two candidates, but the gap is only one percentage point. Which statement best describes this result?
  \begin{enumerate}[label=(\Alph*)]
  \item The difference may lack practical importance despite being statistically significant
  \item Statistical significance guarantees a meaningful real-world effect
  \item The result must be due to a sampling error
  \item Larger sample sizes always make results more practical
  \end{enumerate}
\item If $p=0.60$ and $n=500$, assume the number of supporters follows a binomial model. What is the standard deviation of the count observed in the sample?
  \begin{enumerate}[label=(\Alph*)]
  \item About $11$
  \item About $22$
  \item About $7$
  \item About $3$
  \end{enumerate}
\item According to the slides, what role does the null hypothesis $H_0$ play when analyzing survey data?
  \begin{enumerate}[label=(\Alph*)]
  \item It specifies a benchmark proportion to compare against
  \item It describes the sampling method used
  \item It guarantees the alternative hypothesis is false
  \item It adjusts the confidence level
  \end{enumerate}
\end{enumerate}
\end{document}
