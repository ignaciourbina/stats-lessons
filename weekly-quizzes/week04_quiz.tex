\documentclass{article}
\begin{document}
\begin{enumerate}
\item According to the slides, how is the sample proportion $\hat{p}$ calculated from the number of successes $S$ in a sample of size $n$?
  \begin{enumerate}[label=(\Alph*)]
  \item $\hat{p}=S/n$
  \item $\hat{p}=n/S$
  \item $\hat{p}=S\times n$
  \item $\hat{p}=S-n$
  \end{enumerate}
\item What is the expected value of $\hat{p}$ when the population proportion is $p$?
  \begin{enumerate}[label=(\Alph*)]
  \item $p$
  \item $\hat{p}$
  \item $np$
  \item $p/n$
  \end{enumerate}
\item Which theorem justifies the approximation $\hat{p}\approx N(p,\,p(1-p)/n)$ for large samples?
  \begin{enumerate}[label=(\Alph*)]
  \item Central Limit Theorem
  \item Law of Large Numbers
  \item Bayes' Theorem
  \item Chebyshev's Inequality
  \end{enumerate}
\item The slides describe the significance level $\alpha$ as a tolerance for what type of error when testing hypotheses?
  \begin{enumerate}[label=(\Alph*)]
  \item Type I error
  \item Type II error
  \item Sampling error
  \item Estimation error
  \end{enumerate}
\item What quantity do we obtain by multiplying the estimated standard error by $z_{1-\alpha/2}$ in confidence interval calculations?
  \begin{enumerate}[label=(\Alph*)]
  \item Margin of error
  \item P-value
  \item Test statistic
  \item Residual
  \end{enumerate}
\item A survey records $55$ successes out of $100$ trials. Which of the following gives $\hat{p}$ and its estimated standard error $\sqrt{\hat{p}(1-\hat{p})/n}$?
  \begin{enumerate}[label=(\Alph*)]
  \item $\hat{p}=0.55$,\quad SE $\approx0.05$
  \item $\hat{p}=0.45$,\quad SE $\approx0.05$
  \item $\hat{p}=0.55$,\quad SE $\approx0.15$
  \item $\hat{p}=0.45$,\quad SE $\approx0.15$
  \end{enumerate}
\item Using the values from Question~6, which of the following is the $95\%$ confidence interval for $p$ using $\hat{p}\pm1.96\times\text{SE}$?
  \begin{enumerate}[label=(\Alph*)]
  \item $(0.45,\,0.65)$
  \item $(0.50,\,0.60)$
  \item $(0.30,\,0.80)$
  \item $(0.55,\,0.65)$
  \end{enumerate}
\item A sample of $400$ contains $120$ successes. What is the $99\%$ confidence interval for $p$?
  \begin{enumerate}[label=(\Alph*)]
  \item $(0.24,\,0.36)$
  \item $(0.25,\,0.35)$
  \item $(0.20,\,0.40)$
  \item $(0.30,\,0.40)$
  \end{enumerate}
\item In the voting preference example, $\hat{p}=0.52$, $p_0=0.49$, and $n=1000$. Compute the test statistic $Z$ and decide whether to reject $H_0$ at $\alpha=0.05$.
  \begin{enumerate}[label=(\Alph*)]
  \item $Z\approx1.9$, fail to reject $H_0$
  \item $Z\approx1.9$, reject $H_0$
  \item $Z\approx2.5$, reject $H_0$
  \item $Z\approx0.5$, fail to reject $H_0$
  \end{enumerate}
\item A poll finds $\hat{p}=0.53$ in a sample of $n=1000$. Test $H_0:p=0.50$ against $H_a:p>0.50$ using $Z=(\hat{p}-p_0)/\sqrt{p_0(1-p_0)/n}$. What is the one-sided $P$-value and decision at $\alpha=0.05$?
  \begin{enumerate}[label=(\Alph*)]
  \item $P\approx0.03$, reject $H_0$
  \item $P\approx0.14$, fail to reject $H_0$
  \item $P\approx0.06$, fail to reject $H_0$
  \item $P\approx0.03$, fail to reject $H_0$
  \end{enumerate}
\end{enumerate}
\end{document}
